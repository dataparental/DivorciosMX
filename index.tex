% Options for packages loaded elsewhere
% Options for packages loaded elsewhere
\PassOptionsToPackage{unicode}{hyperref}
\PassOptionsToPackage{hyphens}{url}
\PassOptionsToPackage{dvipsnames,svgnames,x11names}{xcolor}
%
\documentclass[
  letterpaper,
  DIV=11,
  numbers=noendperiod]{scrreprt}
\usepackage{xcolor}
\usepackage{amsmath,amssymb}
\setcounter{secnumdepth}{5}
\usepackage{iftex}
\ifPDFTeX
  \usepackage[T1]{fontenc}
  \usepackage[utf8]{inputenc}
  \usepackage{textcomp} % provide euro and other symbols
\else % if luatex or xetex
  \usepackage{unicode-math} % this also loads fontspec
  \defaultfontfeatures{Scale=MatchLowercase}
  \defaultfontfeatures[\rmfamily]{Ligatures=TeX,Scale=1}
\fi
\usepackage{lmodern}
\ifPDFTeX\else
  % xetex/luatex font selection
\fi
% Use upquote if available, for straight quotes in verbatim environments
\IfFileExists{upquote.sty}{\usepackage{upquote}}{}
\IfFileExists{microtype.sty}{% use microtype if available
  \usepackage[]{microtype}
  \UseMicrotypeSet[protrusion]{basicmath} % disable protrusion for tt fonts
}{}
\makeatletter
\@ifundefined{KOMAClassName}{% if non-KOMA class
  \IfFileExists{parskip.sty}{%
    \usepackage{parskip}
  }{% else
    \setlength{\parindent}{0pt}
    \setlength{\parskip}{6pt plus 2pt minus 1pt}}
}{% if KOMA class
  \KOMAoptions{parskip=half}}
\makeatother
% Make \paragraph and \subparagraph free-standing
\makeatletter
\ifx\paragraph\undefined\else
  \let\oldparagraph\paragraph
  \renewcommand{\paragraph}{
    \@ifstar
      \xxxParagraphStar
      \xxxParagraphNoStar
  }
  \newcommand{\xxxParagraphStar}[1]{\oldparagraph*{#1}\mbox{}}
  \newcommand{\xxxParagraphNoStar}[1]{\oldparagraph{#1}\mbox{}}
\fi
\ifx\subparagraph\undefined\else
  \let\oldsubparagraph\subparagraph
  \renewcommand{\subparagraph}{
    \@ifstar
      \xxxSubParagraphStar
      \xxxSubParagraphNoStar
  }
  \newcommand{\xxxSubParagraphStar}[1]{\oldsubparagraph*{#1}\mbox{}}
  \newcommand{\xxxSubParagraphNoStar}[1]{\oldsubparagraph{#1}\mbox{}}
\fi
\makeatother

\usepackage{color}
\usepackage{fancyvrb}
\newcommand{\VerbBar}{|}
\newcommand{\VERB}{\Verb[commandchars=\\\{\}]}
\DefineVerbatimEnvironment{Highlighting}{Verbatim}{commandchars=\\\{\}}
% Add ',fontsize=\small' for more characters per line
\usepackage{framed}
\definecolor{shadecolor}{RGB}{241,243,245}
\newenvironment{Shaded}{\begin{snugshade}}{\end{snugshade}}
\newcommand{\AlertTok}[1]{\textcolor[rgb]{0.68,0.00,0.00}{#1}}
\newcommand{\AnnotationTok}[1]{\textcolor[rgb]{0.37,0.37,0.37}{#1}}
\newcommand{\AttributeTok}[1]{\textcolor[rgb]{0.40,0.45,0.13}{#1}}
\newcommand{\BaseNTok}[1]{\textcolor[rgb]{0.68,0.00,0.00}{#1}}
\newcommand{\BuiltInTok}[1]{\textcolor[rgb]{0.00,0.23,0.31}{#1}}
\newcommand{\CharTok}[1]{\textcolor[rgb]{0.13,0.47,0.30}{#1}}
\newcommand{\CommentTok}[1]{\textcolor[rgb]{0.37,0.37,0.37}{#1}}
\newcommand{\CommentVarTok}[1]{\textcolor[rgb]{0.37,0.37,0.37}{\textit{#1}}}
\newcommand{\ConstantTok}[1]{\textcolor[rgb]{0.56,0.35,0.01}{#1}}
\newcommand{\ControlFlowTok}[1]{\textcolor[rgb]{0.00,0.23,0.31}{\textbf{#1}}}
\newcommand{\DataTypeTok}[1]{\textcolor[rgb]{0.68,0.00,0.00}{#1}}
\newcommand{\DecValTok}[1]{\textcolor[rgb]{0.68,0.00,0.00}{#1}}
\newcommand{\DocumentationTok}[1]{\textcolor[rgb]{0.37,0.37,0.37}{\textit{#1}}}
\newcommand{\ErrorTok}[1]{\textcolor[rgb]{0.68,0.00,0.00}{#1}}
\newcommand{\ExtensionTok}[1]{\textcolor[rgb]{0.00,0.23,0.31}{#1}}
\newcommand{\FloatTok}[1]{\textcolor[rgb]{0.68,0.00,0.00}{#1}}
\newcommand{\FunctionTok}[1]{\textcolor[rgb]{0.28,0.35,0.67}{#1}}
\newcommand{\ImportTok}[1]{\textcolor[rgb]{0.00,0.46,0.62}{#1}}
\newcommand{\InformationTok}[1]{\textcolor[rgb]{0.37,0.37,0.37}{#1}}
\newcommand{\KeywordTok}[1]{\textcolor[rgb]{0.00,0.23,0.31}{\textbf{#1}}}
\newcommand{\NormalTok}[1]{\textcolor[rgb]{0.00,0.23,0.31}{#1}}
\newcommand{\OperatorTok}[1]{\textcolor[rgb]{0.37,0.37,0.37}{#1}}
\newcommand{\OtherTok}[1]{\textcolor[rgb]{0.00,0.23,0.31}{#1}}
\newcommand{\PreprocessorTok}[1]{\textcolor[rgb]{0.68,0.00,0.00}{#1}}
\newcommand{\RegionMarkerTok}[1]{\textcolor[rgb]{0.00,0.23,0.31}{#1}}
\newcommand{\SpecialCharTok}[1]{\textcolor[rgb]{0.37,0.37,0.37}{#1}}
\newcommand{\SpecialStringTok}[1]{\textcolor[rgb]{0.13,0.47,0.30}{#1}}
\newcommand{\StringTok}[1]{\textcolor[rgb]{0.13,0.47,0.30}{#1}}
\newcommand{\VariableTok}[1]{\textcolor[rgb]{0.07,0.07,0.07}{#1}}
\newcommand{\VerbatimStringTok}[1]{\textcolor[rgb]{0.13,0.47,0.30}{#1}}
\newcommand{\WarningTok}[1]{\textcolor[rgb]{0.37,0.37,0.37}{\textit{#1}}}

\usepackage{longtable,booktabs,array}
\usepackage{calc} % for calculating minipage widths
% Correct order of tables after \paragraph or \subparagraph
\usepackage{etoolbox}
\makeatletter
\patchcmd\longtable{\par}{\if@noskipsec\mbox{}\fi\par}{}{}
\makeatother
% Allow footnotes in longtable head/foot
\IfFileExists{footnotehyper.sty}{\usepackage{footnotehyper}}{\usepackage{footnote}}
\makesavenoteenv{longtable}
\usepackage{graphicx}
\makeatletter
\newsavebox\pandoc@box
\newcommand*\pandocbounded[1]{% scales image to fit in text height/width
  \sbox\pandoc@box{#1}%
  \Gscale@div\@tempa{\textheight}{\dimexpr\ht\pandoc@box+\dp\pandoc@box\relax}%
  \Gscale@div\@tempb{\linewidth}{\wd\pandoc@box}%
  \ifdim\@tempb\p@<\@tempa\p@\let\@tempa\@tempb\fi% select the smaller of both
  \ifdim\@tempa\p@<\p@\scalebox{\@tempa}{\usebox\pandoc@box}%
  \else\usebox{\pandoc@box}%
  \fi%
}
% Set default figure placement to htbp
\def\fps@figure{htbp}
\makeatother





\setlength{\emergencystretch}{3em} % prevent overfull lines

\providecommand{\tightlist}{%
  \setlength{\itemsep}{0pt}\setlength{\parskip}{0pt}}



 


\KOMAoption{captions}{tableheading}
\makeatletter
\@ifpackageloaded{bookmark}{}{\usepackage{bookmark}}
\makeatother
\makeatletter
\@ifpackageloaded{caption}{}{\usepackage{caption}}
\AtBeginDocument{%
\ifdefined\contentsname
  \renewcommand*\contentsname{Table of contents}
\else
  \newcommand\contentsname{Table of contents}
\fi
\ifdefined\listfigurename
  \renewcommand*\listfigurename{List of Figures}
\else
  \newcommand\listfigurename{List of Figures}
\fi
\ifdefined\listtablename
  \renewcommand*\listtablename{List of Tables}
\else
  \newcommand\listtablename{List of Tables}
\fi
\ifdefined\figurename
  \renewcommand*\figurename{Figure}
\else
  \newcommand\figurename{Figure}
\fi
\ifdefined\tablename
  \renewcommand*\tablename{Table}
\else
  \newcommand\tablename{Table}
\fi
}
\@ifpackageloaded{float}{}{\usepackage{float}}
\floatstyle{ruled}
\@ifundefined{c@chapter}{\newfloat{codelisting}{h}{lop}}{\newfloat{codelisting}{h}{lop}[chapter]}
\floatname{codelisting}{Listing}
\newcommand*\listoflistings{\listof{codelisting}{List of Listings}}
\makeatother
\makeatletter
\makeatother
\makeatletter
\@ifpackageloaded{caption}{}{\usepackage{caption}}
\@ifpackageloaded{subcaption}{}{\usepackage{subcaption}}
\makeatother
\usepackage{bookmark}
\IfFileExists{xurl.sty}{\usepackage{xurl}}{} % add URL line breaks if available
\urlstyle{same}
\hypersetup{
  pdftitle={El divorcio en Mexico en datos},
  pdfauthor={Dr.~Fernando Avalos Reyes},
  colorlinks=true,
  linkcolor={blue},
  filecolor={Maroon},
  citecolor={Blue},
  urlcolor={Blue},
  pdfcreator={LaTeX via pandoc}}


\title{El divorcio en Mexico en datos}
\usepackage{etoolbox}
\makeatletter
\providecommand{\subtitle}[1]{% add subtitle to \maketitle
  \apptocmd{\@title}{\par {\large #1 \par}}{}{}
}
\makeatother
\subtitle{Evidencia estadistica y proteccion a la infancia}
\author{Dr.~Fernando Avalos Reyes}
\date{2026-03-10}
\begin{document}
\maketitle

\renewcommand*\contentsname{Table of contents}
{
\hypersetup{linkcolor=}
\setcounter{tocdepth}{2}
\tableofcontents
}

\bookmarksetup{startatroot}

\chapter*{Divorcios en Mexico}\label{divorcios-en-mexico}
\addcontentsline{toc}{chapter}{Divorcios en Mexico}

\markboth{Divorcios en Mexico}{Divorcios en Mexico}

A proposito de las

This is a Quarto book.

To learn more about Quarto books visit
\url{https://quarto.org/docs/books}.

\begin{Shaded}
\begin{Highlighting}[]
\DecValTok{1} \SpecialCharTok{+} \DecValTok{1}
\end{Highlighting}
\end{Shaded}

\begin{verbatim}
[1] 2
\end{verbatim}

\bookmarksetup{startatroot}

\chapter{Introduccion: El divorcio es un fenomeno de salud publica y
decision
informada}\label{introduccion-el-divorcio-es-un-fenomeno-de-salud-publica-y-decision-informada}

\bookmarksetup{startatroot}

\chapter{1.1 El divorcio como indicador de salud
social}\label{el-divorcio-como-indicador-de-salud-social}

El divorcio no es solo un acto jurídico que disuelve un matrimonio: es
un \textbf{indicador sensible del bienestar social}, íntimamente ligado
a la salud mental, la economía doméstica, la equidad de género y la
estabilidad de la infancia.\\
En México, su análisis desde los datos no solo revela transformaciones
culturales, sino también \textbf{desigualdades estructurales} en el
acceso a la justicia, el cuidado infantil y la protección económica.

La separación conyugal, cuando no se gestiona adecuadamente, puede
generar \textbf{efectos adversos en la salud física y emocional} de los
involucrados, especialmente de niñas, niños y adolescentes. Por ello,
este libro propone observar el divorcio como un \textbf{evento con
consecuencias sanitarias y sociales}, que demanda estrategias de
prevención, acompañamiento y evaluación desde el sistema público.

\bookmarksetup{startatroot}

\chapter{1.2 El vacío entre la evidencia y la decisión
pública}\label{el-vacuxedo-entre-la-evidencia-y-la-decisiuxf3n-puxfablica}

A pesar de la abundancia de registros administrativos ---actas de
divorcio, juicios familiares, censos y encuestas---, México carece de
una \textbf{infraestructura integrada de datos familiares} que permita
diseñar políticas basadas en evidencia.

\hfill\break
Las decisiones sobre custodia, pensión alimenticia o convivencia se
toman, en muchos casos, sin respaldo estadístico ni análisis de impacto
longitudinal.

El divorcio, al igual que otras transiciones familiares, \textbf{afecta
la salud mental, el acceso a servicios, la movilidad económica y la
educación}. No obstante, la información sobre estos efectos se encuentra
dispersa entre instituciones, sin interoperabilidad ni criterios
unificados.

Este libro busca \textbf{cerrar esa brecha entre evidencia y decisión},
ofreciendo un análisis técnico y visualmente accesible que vincula los
datos judiciales, sanitarios y sociodemográficos en torno al divorcio en
México.

\bookmarksetup{startatroot}

\chapter{1.3 Objetivos del libro}\label{objetivos-del-libro}

El propósito de esta obra es aportar herramientas concretas para la
\textbf{toma de decisiones informadas en políticas públicas}, mediante
un análisis reproducible y transparente.\\
Se persiguen tres objetivos centrales:

\begin{enumerate}
\def\labelenumi{\arabic{enumi}.}
\tightlist
\item
  \textbf{Analizar los datos nacionales y estatales} sobre divorcio,
  custodia y pensión desde una perspectiva de salud pública.\\
\item
  \textbf{Identificar patrones y desigualdades} que afectan la
  protección de la infancia y el bienestar familiar.\\
\item
  \textbf{Proponer líneas de política pública y gestión
  interinstitucional} sustentadas en evidencia empírica.
\end{enumerate}

\bookmarksetup{startatroot}

\chapter{1.4 Perspectiva metodológica y de
datos}\label{perspectiva-metodoluxf3gica-y-de-datos}

El libro adopta un \textbf{enfoque de datos abiertos y reproducibilidad
científica}.\\
Las bases analizadas provienen de fuentes como:

\begin{itemize}
\tightlist
\item
  \textbf{INEGI}: estadísticas vitales y registros administrativos.\\
\item
  \textbf{Consejos de la Judicatura estatales}: sentencias y
  resoluciones de divorcio.\\
\item
  \textbf{Secretaría de Salud y CONAPO}: indicadores de salud mental y
  demografía.\\
\item
  \textbf{Programas estatales de protección a la infancia y familia
  (DIF, SIPINNA).}
\end{itemize}

El análisis combina métodos cuantitativos (series de tiempo,
georreferenciación, modelado predictivo) con un componente cualitativo
orientado a la \textbf{interpretación de los efectos del divorcio en la
salud familiar y social}.

\bookmarksetup{startatroot}

\chapter{1.5 Estructura de la obra}\label{estructura-de-la-obra}

El contenido se organiza en torno a los ejes de evidencia, análisis y
decisión pública:

\begin{enumerate}
\def\labelenumi{\arabic{enumi}.}
\tightlist
\item
  \textbf{Marco jurídico y conceptual:} evolución legal del divorcio y
  su traducción en política pública.\\
\item
  \textbf{Fuentes y metodología:} integración de bases judiciales,
  demográficas y sanitarias.\\
\item
  \textbf{Tendencias nacionales:} análisis temporal y espacial del
  divorcio en México.\\
\item
  \textbf{Custodia y protección infantil:} datos sobre guarda,
  convivencia y efectos en salud mental.\\
\item
  \textbf{Pensión alimenticia y bienestar económico:} distribución,
  cumplimiento y desigualdades.\\
\item
  \textbf{Impactos en salud pública:} depresión, violencia
  intrafamiliar, y riesgo psicosocial post-divorcio.\\
\item
  \textbf{Política pública basada en datos:} propuestas para fortalecer
  la evidencia institucional.\\
\item
  \textbf{Conclusiones y recomendaciones:} estrategias para articular
  justicia, salud y bienestar familiar.
\end{enumerate}

\bookmarksetup{startatroot}

\chapter{1.6 Hacia un modelo de política pública basada en
evidencia}\label{hacia-un-modelo-de-poluxedtica-puxfablica-basada-en-evidencia}

El divorcio es un espejo de las relaciones sociales, pero también un
campo donde convergen \textbf{justicia, salud y gobernanza de datos}.\\
México cuenta hoy con las condiciones técnicas y legales para transitar
hacia una \textbf{política pública familiar basada en datos}, que
permita diseñar intervenciones preventivas, priorizar recursos y evaluar
resultados con rigor científico.

Este libro no pretende dictar verdades jurídicas, sino \textbf{ofrecer
evidencia verificable} para que tomadores de decisiones, legisladores,
investigadores y organismos de salud puedan \textbf{construir políticas
más humanas, eficientes y centradas en la infancia}.

\begin{center}\rule{0.5\linewidth}{0.5pt}\end{center}

\begin{quote}
\emph{``Donde hay datos, hay justicia posible; donde no los hay, solo
hay suposiciones.''}\\
--- Adaptación de Florence Nightingale, pionera de la estadística
sanitaria.
\end{quote}

\bookmarksetup{startatroot}

\chapter{Summary}\label{summary}

In summary, this book has no content whatsoever.

\begin{Shaded}
\begin{Highlighting}[]
\DecValTok{1} \SpecialCharTok{+} \DecValTok{1}
\end{Highlighting}
\end{Shaded}

\begin{verbatim}
[1] 2
\end{verbatim}

\bookmarksetup{startatroot}

\chapter{Summary}\label{summary-1}

In summary, this book has no content whatsoever.

\begin{Shaded}
\begin{Highlighting}[]
\DecValTok{1} \SpecialCharTok{+} \DecValTok{1}
\end{Highlighting}
\end{Shaded}

\begin{verbatim}
[1] 2
\end{verbatim}

\bookmarksetup{startatroot}

\chapter{Summary}\label{summary-2}

In summary, this book has no content whatsoever.

\begin{Shaded}
\begin{Highlighting}[]
\DecValTok{1} \SpecialCharTok{+} \DecValTok{1}
\end{Highlighting}
\end{Shaded}

\begin{verbatim}
[1] 2
\end{verbatim}

\bookmarksetup{startatroot}

\chapter{Summary}\label{summary-3}

In summary, this book has no content whatsoever.

\begin{Shaded}
\begin{Highlighting}[]
\DecValTok{1} \SpecialCharTok{+} \DecValTok{1}
\end{Highlighting}
\end{Shaded}

\begin{verbatim}
[1] 2
\end{verbatim}

\bookmarksetup{startatroot}

\chapter*{References}\label{references}
\addcontentsline{toc}{chapter}{References}

\markboth{References}{References}

\phantomsection\label{refs}




\end{document}
